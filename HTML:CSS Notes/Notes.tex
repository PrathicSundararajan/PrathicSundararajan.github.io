\documentclass{article}
\usepackage[]{mathtools}
\begin{document}
\title{HTML/CSS Notes v2}
\author{Prathic Sundararajan}
\date{}
\maketitle
\section{HTML}
	\subsection{Different Containers/Useful Tags}
	\begin{description}
		\item[a)] div - This is often used with css elements
		\begin{description}
			\item Property - "word-wrap: break-word"
		\end{description}						
		\item[b)]  br -  Can be used to add a line break between containers and  w/o any close tag
		\item[c)] img - used to show images
		\begin{description}
			\item TagLine- src= "linkToURL"
			\item Property- min-width: 25%
			\item Property- max-width: 75%
			\item Property- border-radius: 50%
			\begin{description}
				\item This can be used to cut the picture into an oval
			\end{description}
		\end{description}	
	\end{description}
	\subsection{Useful Extra Stuff/Properties}
		\begin{description}
		\item[a)] Flexed Container - This can be used to have things side by side
		\begin{description}
			\item Refer to code to see the needed CSS formatting as well as how to format in html
		\end{description}
		\end{description}
\section{CSS}
	\subsection{Selectors}
		\begin{description}
		\item[a)] Type Selectors
		\item[b)] Class Selectors
		\item[c)] ID Selectors
		\item[d)] Additional Selectors
		\end{description}
	\subsection{Main Different Components}
		\begin{description}
		\item[a)] Top Nav Bar
			\begin{description}
			\item This is used to make a navigation bar at the top of the page
			\end{description}
		\item[b)] Flex Box
			\begin{description}
			\item This was very useful to have items side by side. Need to use max heights and width. Border radius allows you to cut up the picture into an oval
			\end{description}
		\end{description}
\section{Interfacing}
\end{document}


<font color="white" id = "name"> Introduction Pages! </font>